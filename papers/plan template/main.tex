\documentclass[english, a4paper]{article}
\usepackage[T1]{fontenc}
\usepackage[latin9]{inputenc}
\usepackage{geometry}
\usepackage{url}
\usepackage{babel}
\usepackage{csquotes}
\usepackage[]{biblatex}
\addbibresource{bibliography.bib}
\geometry{verbose,tmargin=3cm,bmargin=3cm,lmargin=3cm,rmargin=3cm}

\title{
    Research Plan for CSE3000 Research Project\\
    {\Large \textit{(Temporary) Title of Your Project}}
}
\author{Your Name}

\begin{document}
\maketitle


\section*{Background of the research}
In this section you should give some background of your research area. What is the problem you are tackling? Why is it worth to solve it? Who has done some work in this area? What have they achieved? Include references to previous work. Also, describe what is still missing (the knowledge gap) -- this may be much more complex and further into the future than your planned contribution.


\section*{Research Question}
Explicitly state the main question you aim to answer or the hypothesis you aim to test. Argue why it is reasonable to achieve in the time frame of your project and make references to previous work that backs your arguments. Explain what you expect to accomplish by working on this project.  

Break down the main question(s) into sub-questions -- this will help you put more structure into your research. Aim to formulate (sub-)questions that are sufficiently concrete, such that other students would be able to answer them with a single experiment or proof, and that you would be able to judge whether they have done well. For instance, it is better to say: "under which conditions (e.g. problem size) is method A better than method B?" than "How can you do better than method B?". Include objective criteria for success. For instance, ``lower runtime than algorithm A'', ``better quality than method B'' or ``using fewer data samples than C''. Try to imagine how you would present the ideal outcome. For instance, a scatter plot or a histogram.

\section*{Method}
In this section you should outline how you intend to go
about accomplishing the aims you have set in the previous
section. Try to break your aims down into small, achievable tasks. 
Which tools, software, or data are you going to use? With whom do you intend to collaborate on what (if anyone)? What are their tasks? What are your tasks? Are there any dependencies between these tasks? Also, make sure to think how you will decide if a task has been completed successfully.

\section*{Planning of the research project}
Try to estimate how long you will spend on each task, and create a timetable for each sub-task. We do not expect you to provide an exact date for every single activity but you should indicate your general planning for this quarter. Include all of the course deadlines (midterm presentation, paper draft v1 for peer review and supervisor feedback, paper draft v2 for feedback from responsible professor, etc.) from Brightspace. If you haven't done so yet, this is a good opportunity to look through all assignments and the Research Project schedule. Make sure to consider the time for writing and don't postpone this to the latest moment; we suggest that you introduce for yourself personal deadlines when you aim to complete every section. This assignment aims to help you organize your work and stay on track -- you may find it useful to print your schedule. Structure this section as a table with every important date; you should at least including the following:

\begin{enumerate}
\item milestones for completion of the identified tasks and sub-tasks;
\item meetings with your peer group;
\item meetings with your supervisor (these may overlap with the above) -- you can already think what may be important discuss given your schedule
\item meetings with your responsible professor (these may overlap with the above)  -- again, think what you would like to discuss;
\item deadlines from the course manual on Brightspace;
\item final presentation for your supervising team.
\end{enumerate}

\noindent Consider organizing the activities per week or labeling them (course task, project task, meeting, \ldots). As a rough estimate, you should arrive at a total of 30-50 activities for the whole quarter.

\printbibliography 

\end{document}
